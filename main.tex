\documentclass[12pt]{book}
%\usepackage[UTF8]{ctex}%
\linespread{1.5}%1.5倍行距
%\usepackage[no-math]{fontspec}
\usepackage{fontspec}
%\usepackage{unicode-math}
\usepackage{amssymb}%与unicode-math冲突
\usepackage{arev}%与可爱字体搭配的数学字体
\setmainfont{Comic Sans MS}
%\setmainfont{Segoe Print}
%\setmathfont{TeX Gyre Bonum Math}
%\usepackage{comicsans}
\usepackage{amsthm}
\usepackage{amsmath}
\usepackage{amscd}
\usepackage{amsfonts}
\usepackage{amstext}
\usepackage{array}%数组和表格制作
\usepackage{caption}
\usepackage{color}
\usepackage{commutative-diagrams}%用CoDi包画交换图
\usepackage{epsfig}
\usepackage{graphicx}%插图
\usepackage{indentfirst}%首行缩进宏包
\usepackage{latexsym}
\usepackage{listings}
\usepackage{multicol}%跨列表格
\usepackage{multirow}%跨行表格
\usepackage{rawfonts}
\usepackage{tabularx}%自动设置表格列宽
\usepackage{tikz}
\usepackage{tikz-cd}%tikzcd画交换图
\usepackage{titletoc}%目录格式设置
\usepackage[all]{xy}

\usepackage{geometry}
\geometry{
a4paper,
left = 2.5cm,
right = 2.5cm,
top = 3cm,
bottom = 3cm,
}
\usepackage{hyperref}%超链接
\hypersetup{colorlinks, linkcolor={blue}, citecolor={blue}} 
\usepackage{mathrsfs}%数学花体字
\usepackage{titlesec}
% \renewcommand{\thesection}{\thechapter\alph{section})}
% \titleformat{\section}[runin]{\normalfont\bfseries}{\indent\thesection\ }{0pt}{}
\usepackage{xcolor}
\definecolor{peach1}{rgb}{0.2902,0.5294,0.6314}%水蜜桃汽水配色1
\definecolor{peach2}{rgb}{0.8706,0.5647,0.4824}%水蜜桃汽水配色2
\definecolor{peach3}{rgb}{0.9020,0.7725,0.7059}%水蜜桃汽水配色3
\definecolor{prin}{rgb}{0.4588,0.6980,0.9059}%公主配色蓝
\definecolor{straw}{rgb}{0.7608,0.5216,0.5412}%莓果配色红
\newcounter{acounter}[chapter]
\numberwithin{acounter}{chapter}
\newtheoremstyle{theo}% name
{4pt}% Space above1
{4pt}% Space below1
{\normalfont }% Body font
{}% Indent amount2
{\color{peach1}\bfseries }% Theorem head font
{.}% Punctuation after theorem head
{.5em}% Space after theorem head3
{}% Theorem head spec (can be left empty, meaning ‘normal’)
\newtheoremstyle{defi}% name
{4pt}% Space above1
{4pt}% Space below1
{\normalfont }% Body font
{}% Indent amount2
{\color{peach2}\bfseries }% Theorem head font
{.}% Punctuation after theorem head
{.5em}% Space after theorem head3
{}% Theorem head spec (can be left empty, meaning ‘normal’)

\makeatletter
\renewenvironment{proof}[1][\proofname]{\par
\pushQED{\qed}%
\normalfont \topsep6\p@\@plus6\p@\relax
\trivlist
\item\relax
{\color{prin}\normalfont\bfseries
#1\@addpunct{.}}\hspace\labelsep\ignorespaces
}{%
\popQED\endtrivlist\@endpefalse
}
\makeatother

{\theoremstyle{theo}
%\swapnumbers
\newtheorem{thm}[acounter]{Theorem}%定理
\newtheorem{prop}[acounter]{Proposition}%命题
\newtheorem{cor}[acounter]{Corollary}%推论
\newtheorem{lem}[acounter]{Lemma}%引理
\newtheorem{rem}[acounter]{Remark}%注
\newtheorem{eg}[acounter]{Example}%例
\newtheorem{ex}[acounter]{Exercise}%习题
}
{\theoremstyle{defi}
%\swapnumbers
\newtheorem{defn}[acounter]{Definition}%定义
}

\newcommand{\tfa}[1]{{\color{peach1}\textbf{#1}}}%水蜜桃色1
\newcommand{\tfb}[1]{{\color{peach2}\textbf{#1}}}%水蜜桃色2
\newcommand{\tfc}[1]{{\color{peach3}\textbf{#1}}}%水蜜桃色3
\newcommand{\tfs}[1]{{\color{straw}\textbf{#1}}}%莓果配色红
\newcommand{\diff}{\mathrm{d}}

\newcommand{\ad}{\operatorname{ad}}
\renewcommand{\sl}{\mathrm{sl}}
\newcommand{\gl}{\mathrm{gl}}
\newcommand{\ima}{\operatorname{im}}
\newcommand{\Der}{\operatorname{Der}}
\newcommand{\tr}{\operatorname{tr}}
\newcommand{\End}{\operatorname{End}}
\renewcommand{\char}{\operatorname{char}}
\renewcommand{\span}{\operatorname{span}}
\newcommand{\rad}{\operatorname{rad}}

\usepackage{dsfont}
\DeclareMathOperator{\id}{\mathds{1}}

\usepackage[inline]{enumitem}

\renewcommand{\proofname}{Proof}%
\renewcommand{\bibname}{Bibliography}%
\renewcommand{\contentsname}{Contents}%

\title{\textbf{Notes on Lie Algebra}}
\author{Tom}
\date{\today}

\everymath{\displaystyle}

\begin{document}
\frontmatter
\maketitle
\tableofcontents
\mainmatter
\chapter{Introduction}

Some of the notes come from the lectures given by Professor Chongying Dong in the summer school of Xiamen University in 2019. Other references of the notes are \cite{Humphreys_Lie_alg,Intro_to_Lie_alg}. Most notations in this notes are consistent with \cite{Intro_to_Lie_alg}.

Our goal is to classify all the finite-dimensional complex semisimple Lie algebras up to isomorphism. Unless otherwise stated, all the Lie algebras we discuss are finite-dimensional.

\section{Basic Concepts}

In this section we introduce some basic concepts about Lie algebras and their ideals. We will omit some easy proofs.

\begin{defn}
    A \tfs{Lie algebra} $L$ is a vector space over a field $F$ with a bilinear product $L\times L\to L$, $(x,y)\mapsto [x,y]$, called the Lie bracket, satisfying the following properties:
    \begin{enumerate}
        \item $[x,x]=0$, $\forall x\in L$;
        \item $[x,[y,z]]+[y,[z,x]]+[z,[x,y]]=0$, $\forall x,y,z\in L$, called the \tfs{Jacobi identity}.
    \end{enumerate}
\end{defn}

\begin{rem}
    The condition 1 implies $[x,y]=-[y,x]$, $\forall x,y\in L$, and the converse is also true when $\char F\neq 2$.
\end{rem}

\begin{defn}
    A \tfs{Lie subalgebra} $K$ of a given Lie algebra $L$ is a vector subspace of $L$ such that $[x,y]\in K$ for all $x,y\in K$.
\end{defn}

Here are some examples of Lie algebras.
\begin{enumerate}
    \item Let $\gl(n,F):=M_{n\times n}(F)$ be the vector space of all $n\times n$ matrices over the field $F$ with the Lie bracket defined by $[x,y]:=xy-yx$, where $xy$ is the usual product of matrices $x$ and $y$. Then $\gl(n,F)$ is a Lie algebra, called the \tfs{general linear algebra}.
    \item Let $V$ be a vector space over a field $F$. Then $\gl(V):=\operatorname{End}(V)$ with the Lie bracket $[x,y]:=x\circ y-y\circ x$ for $x,y\in\gl(V)$ is a Lie algebra, also known as the \tfs{general linear algebra}.
    \item Let $V$ be a vector space with the Lie bracket defined by $[x,y]=0$ for all $x,y\in L$. Then this is an \tfs{abelian} Lie algebra structure on $V$.
    \item Let $\sl(n,F)$ be the subspace of $\gl(n,F)$ consisting of all matrices of trace 0. Then $\sl(n,F)$ is a Lie algebra with the same Lie bracket as $\gl(n,F)$, called the \tfs{special linear algebra}.
    \item Let $\mathrm{b}(n,F)$ be the upper triangular matrices in $\gl(n,F)$. Then $\mathrm{b}(n,F)$ is a Lie algebra with the same Lie bracket as $\gl(n,F)$.
    \item Similarly, let $\mathrm{n}(n,F)$ be the strictly upper triangular matrices in $\gl(n,F)$. Then $\mathrm{n}(n,F)$ is a Lie algebra with the same Lie bracket as $\gl(n,F)$.
    \item Let $A$ be an algebra over a field $F$ (i.e. $A$ is both a ring and a vector space over $F$). Let
    \[
        \Der(A):=\{ D\in\operatorname{End}(A):D(ab)=D(a)b+aD(b) \mbox{ for all } a,b\in A \}.
    \]
    Any element in $\Der(A)$ is called a \tfs{derivation} of $A$. We can verify that if $D,E\in\Der A$, then $[D,E]=D\circ E-E\circ D \in \Der A$ (but $D \circ E$ may not be a derivation). Then $\Der(A)$ is a Lie subalgebra of $\gl(A)$.
\end{enumerate}


\begin{defn}
    Let $L$ be a Lie algebra. An \tfs{ideal} of $L$ is a subspace $I$ of $L$ such that $[x,y]\in L$ for all $x\in I,y\in L$ (or equivalently, for all $x\in L,y\in I$).
\end{defn}

An ideal is always a subalgebra, but a subalgebra need not be an ideal. For example, $\mathrm{b}(n,F)$ is a subalgebra of $\gl(n,F)$ but not an ideal when $n\geq 2$.

The Lie algebra $L$ itself is an ideal of $L$, and $\{0\}$ is an ideal of $L$. They are called the \tfs{trivial ideals} of $L$. Another example of an ideal is the \tfs{center} $Z(L)$ of $L$, defined by 
\[
    Z(L):=\{x\in L:[x,y]=0 \mbox{ for all } y\in L\}.
\]
Clearly $Z(L)=L$ if and only if $L$ is abelian.

Now we introduce the definition of homomorphisms between Lie algebras.
\begin{defn}
    Let $L_1$, $L_2$ be two Lie algebras over a field $F$. We say a map $\varphi:L_1\to L_2$ is a \tfs{homomorphism}, if $\varphi$ is a linear map, and 
    \[
        \varphi([x,y])=[\varphi(x),\varphi(y)],\quad \forall x,y\in L.
    \]
    Notice that the first Lie bracket is taken in $L_1$ and the second is taken in $L_2$. We say $\varphi$ is an \tfs{isomorphism} if $\varphi$ is also bijective.
\end{defn}

An extremely important homomorphism is the \tfs{adjoint homomorphism}. If $L$ is a Lie algebra, then we define 
\[
    \ad:L\to \gl(L)
\]
by $(\ad x)(y):=[x,y]$ for $x,y\in L$. Clearly $\ad$ is a linear map, and by Jacobi identity we can check that
\[
    \ad([x,y])=\ad x\circ \ad y-\ad y\circ \ad x=[\ad x,\ad y].
\]
Hence $\ad:L\to \gl(L)$ is a Lie algebra homomorphism. Its kernel is $Z(L)$.

\begin{prop}
    If $\varphi : L_1\to L_2$ is a Lie algebra homomorphism, then $\ker \varphi$ is an ideal and $\ima \varphi$ is a subalgebra.
\end{prop}

\begin{proof}
    This is easy to verify.
\end{proof}

If $I,J$ are ideals of the Lie algebra $L$, then their intersection $I\cap J$, their sum $I+J:=\{x+y:x\in I,y\in J\}$ and their Lie bracket $[I,J]:=\operatorname{Span}\{[x,y]:x\in I,y\in J\}$ are ideals of $L$.

\begin{rem}
    It is necessary to define $[I,J]$ to be the \tfs{span} of commutators $[x,y]$ rather than just the \tfs{set} of commutators. Sometimes the set of commutators is not an ideal. See \cite[Exercise 2.14]{Intro_to_Lie_alg} for an example.
\end{rem}

We write $L':=[L,L]$, called the \tfs{derived algebra} of $L$.

If $I$ is an ideal of the Lie algebra $L$, then the cosets $z+I=\{z+x:x\in I\}$ for $z\in L$ form a quotient vector space $L/I=\{z+I:z\in L\}$. A Lie bracket on $L/I$ can be defined by $[w+I,z+I]:=[w,z]+I$ for $w,z\in L$. We can verify that the Lie bracket on $L/I$ is well-defined, which makes $L/I$ a Lie algebra, called the \tfs{quotient algebra} of $L$ by $I$.


\begin{thm}
    We have some isomorphism theorems for Lie algebras:
    \begin{enumerate}
        \item If $\varphi:L_1\to L_2$ is a homomorphism of Lie algebras, then 
        \[
            L_1/\ker \varphi\cong \ima \varphi.
        \]
        \item If $I$ and $J$ are ideals of a Lie algebra, then $(I+J)/J\cong I/(I\cap J)$.
        \item If $I$ and $J$ are ideals of a Lie algebra $L$ and $I\subset J$, then $J/I$ is an ideal of $L/I$ and $(L/I)/(J/I)\cong L/J$.
    \end{enumerate}
\end{thm}

\begin{proof}
    This is easy to verify.
\end{proof}

\begin{eg}
    Fix a field $F$ and consider the linear map $\operatorname{tr}:\gl (n,F)\to F$, which sends a matrix to its trace. It is easy to see that $\tr$ is a surjective Lie algebra homomorphism, and $\ker\tr =\sl(n,F)$. Then $\gl(n,F)/\sl(n,F)\cong F$.
\end{eg}

\begin{eg}
    Recall that for a Lie algebra $L$, the kernel of the adjoint homomorphism $\ad:L\to \gl (L)$ is the center $Z(L)$. If we denote $\ad L:=\ima \ad$, then $L/Z(L)\cong \ad L$, which is a Lie subalgebra of $\gl(L)$.
\end{eg}

Let $I$ be an ideal of the Lie algebra $L$. There is a bijective correspondence between the ideals of $L/I$ and the ideals of $L$ containing $I$. If $J$ is an ideal of $L$ that contains $I$, then $J/I$ is an ideal of $L/I$. Conversely, if $K$ is an ideal of $L/I$, then $J:=\{z\in L:z+I\in K\}$ is an ideal of $L$ that contains $I$.

For two Lie algebras $L_1,L_2$, we define their \tfs{direct sum} $L_1\oplus L_2:=\{(x_1,x_2):x_i\in L_i\}$ to be the direct sum of their underlying vector spaces with the Lie bracket defined by 
\[
    [(x_1,x_2),(y_1,y_2)]:=([x_1,y_1],[x_2,y_2]).
\] 
Then $L_1\oplus L_2$ is a Lie algebra.

Next we introduce the notion of solvable and nilpotent Lie algebras. 

\begin{lem}
    If $I$ is an ideal of the Lie algebra $L$. Then $L/I$ is abelian if and only if $I$ contains the derived algebra $L'$.
\end{lem}

\begin{proof}
    Clear.
\end{proof}

This lemma tells us that the derived algebra is the smallest ideal of $L$ with an abelian quotient. We define the derived series of $L$ to be the series with terms
\[
    L^{(1)}=L' \quad \mbox{and} \quad L^{(k)}=[L^{(k-1)},L^{(k-1)}] \mbox{ for } k\geq 2.
\]

Then $L\supset L^{(1)}\supset L^{(2)}\supset\cdots$. Each $L^{(k)}$ is an ideal of $L$.

\begin{defn}
    A Lie algebra $L$ is called \tfs{solvable} if for some $m\geq 1$ we have $L^{(m)}=0$.
\end{defn}

\begin{lem}
    Let $L$ be a Lie algebra.
    \begin{enumerate}
        \item If $L$ is solvable, then every subalgebra and every homomorphic image of $L$ are solvable.
        \item If $I$ is an ideal of $L$ such that both $I$ and $L/I$ are solvable, then $L$ is solvable.
        \item If $I$ and $J$ are solvable ideals of $L$, then $I+J$ is also a solvable ideal of $L$.
    \end{enumerate}
\end{lem}

\begin{proof}
    This is easy to verify.
\end{proof}

\begin{cor}
    Let $L$ be a finite dimensional Lie algebra. Then there exists a unique maximal solvable ideal containing every solvable ideal.
\end{cor}

\begin{proof}
    Let $R$ be a solvable ideal of $L$ with the largest possible dimension. Then for every solvable ideal $I$ of $L$, $R+I$ is also a solvable ideal, and $\dim (R+I)\geq\dim R$. Hence $R+I=R$, so $I\subset R$.
\end{proof}

The unique maximal solvable ideal above is called the \tfs{radical} of $L$, and is denoted by $\rad L$.

\begin{defn}
    A non-zero finite dimensional Lie algebra $L$ is called \tfs{semisimple} if $\rad L=0$.
\end{defn}

\begin{prop}
    A Lie algebra is semisimple if and only if every abelian ideal is 0. (Recall that an ideal $I$ of a Lie algebra is called abelian if $[I,I]=0$.)
\end{prop}

\begin{proof}
    Easy.
\end{proof}

\begin{defn}
    A finite dimensional Lie algebra $L$ is called \tfs{simple} if it has no ideals other than 0 and $L$, and $L$ is not abelian.
\end{defn}

\begin{rem}
    Clearly simple Lie algebras are semisimple. The requirement that A simple Lie algebra should not be abelian is to remove the 1-dimensional Lie algebra.
\end{rem}

We will see later that every semisimple Lie algebra is a direct sum of simple Lie algebras, which justifies the name \tfs{semi}simple.

\begin{lem}
    If $L$ is a Lie algebra, then $L/\rad L$ is semisimple.
\end{lem}

\begin{proof}
    Easy.
\end{proof}

This lemma tells us that if we want to understand finite dimensional Lie algebras, it is necessary to understand solvable Lie algebras and semisimple Lie algebras.

Next we introduce the definition of a nilpotent Lie algebra, which is also a very important kind of Lie algebras.

\begin{defn}
    Let $L$ be a Lie algebra. The lower central series of $L$ is defined to be the series with terms 
    \[
        L^{1}=L' \quad \mbox{and} \quad L^{k}=[L,L^{k-1}] \mbox{ for } k\geq 2.
    \]
    Then $L\supset L^{1}\supset L^{2}\supset\cdots$. $L$ is called \tfs{nilpotent} if for some $m\geq 1$ we have $L^m=0$.
\end{defn}

Clearly, nilpotent Lie algebras are solvable, but solvable Lie algebras may not be nilpotent.

\begin{prop}\label{adL_nilpotent_iff_L_is}
    Let $L$ be a Lie algebra.
    \begin{enumerate}
        \item If $L$ is nilpotent, then every subalgebra of $L$ is nilpotent.
        \item $L$ is nilpotent if and only if $\ad L\cong L/Z(L)$ is nilpotent.
    \end{enumerate}
\end{prop}

\begin{proof}
    Easy.
\end{proof}

\section{A Little Representation Theory}

Let $L$ be a Lie algebra over a field $F$. A \tfs{representation} of $L$ is a Lie algebra homomorphism $\varphi:L\to \gl(V)$, where $V$ is a finite-dimensional vector space over $F$. If we fix a basis of $V$, then we obtain a homomorphism $L\to \gl(n,F)$, called the matrix representation of $L$. If $\ker \varphi=0$, then the representation is said to be faithful.

\begin{eg}
    The \tfs{adjoint representation} of a Lie algebra $L$ is 
    \[
        \ad:L\to \gl(L),\quad x\mapsto \ad x.
    \]
    The kernel of the adjoint representation is $Z(L)$.
\end{eg}

\begin{defn}
    Suppose $L$ is a Lie algebra over a field $F$. An $L$-\tfs{module} is a finite-dimensional $F$-vector space $V$, together with a map 
    \[
        L\times V\to V,\quad (x,v)\mapsto x \cdot v
    \]
    satisfying the following properties:
    \[
        (\lambda x+\mu y)\cdot v=\lambda(x\cdot v)+\mu(y\cdot v),\eqno(1)
    \]
    \[
        x\cdot (\lambda v+\mu w)=\lambda(x\cdot v)+\mu(x\cdot w),\eqno(2)
    \]
    \[
        [x,y]\cdot v=x\cdot (y\cdot v)-y\cdot (x\cdot v),\eqno(3)
    \]
    for all $x,y\in L$, $v,w\in V$, and $\lambda,\mu\in F$.
\end{defn}

The representations of a Lie algebra $L$ and $L$-modules are two different ways to describe the same structures.

Let $L$ be a Lie algebra. Suppose $V$ is an $L$-module. A \tfs{submodule} $W$ of $V$ is a subspace of $V$ which is invariant under the action of $L$. 

\begin{eg}
    A Lie algebra $L$ can be viewed as an $L$-module via the adjoint representation. Then the submodules of $L$ are exactly the ideals of $L$.
\end{eg}

If $W$ is a submodule of the $L$-module $V$, then the quotient space $V/W$ is naturally an $L$-module, called the \tfs{quotient module}.

Now we introduce the homomorphism between $L$-modules. Suppose $V,W$ are $L$-modules. An $L$-module \tfs{homomorphism} from $V$ to $W$ is a linear map $\theta:V\to W$ which commutes with the action of $L$. We say $\theta$ is an \tfs{isomorphism} if it is bijective.

The classical isomorphism theorems for usual modules (over a ring) also hold for $L$-modules.

Let $V$ be an $L$-module. We say $V$ is \tfs{irreducible} if it is non-zero and has no submodules other than 0 and $V$. We say $V$ is \tfs{indecomposable} if $V$ cannot be written as the direct sum of two non-zero submodules. We say $V$ is \tfs{completely reducible} if $V$ can be written as the direct sum of some irreducible submodules.

\begin{lem}[Schur's Lemma]
    Let $L$ be a complex Lie algebra. If $S$ is a finite-dimensional irreducible $L$-module, then every $L$-module homomorphism $\theta:S\to S$ is a scalar multiple of the identity transformation, i.e. $\theta=\lambda\cdot\id_S$ for some $\lambda\in\mathbb{C}$.
\end{lem}

\begin{proof}
    Note that $\theta$ must have an eigenvalue, say $\lambda$. Then $\ker (\theta-\lambda\cdot\id_S)$ is a non-zero submodule of $S$, which must be the whole $S$. This completes the proof.
\end{proof}

\begin{thm}[Weyl's Theorem]\label{weyl's_thm}
    Let $L$ be a complex semisimple Lie algebra. Then every irreducible representation of $L$ is completely reducible.
\end{thm}

\begin{proof}
    See \cite[Appendix B]{Intro_to_Lie_alg} or \cite[p. 28]{Humphreys_Lie_alg}.
\end{proof}















\chapter{Some Basic Theorems}
We have shown that for any finite-dimensional Lie algebra $L$, its radical $\rad L$ is solvable, and the quotient $L/\rad L$ is semisimple. So from now on, we mainly focus on solvable Lie algebras and semisimple Lie algebras.

To understand a Lie algebra $L$, we usually consider its representation, i.e. a Lie algebra homomorphism $L\to \gl(V)$ for some vector space $V$. If the representation is faithful, then no information about $L$ will be lost. So we usually assume that $L$ is a subalgebra of $\gl(V)$ for some vector space $V$. Although it is a theorem (Ado's Theorem) that every Lie algebra has a faithful representation, we shall not prove it.

\section{Engel's Theorem}

Engel's Theorem is the first non-trivial theorem that we will come across. It is a very useful tool in the study of Lie algebras.

\begin{thm}[Engel's Theorem]
    Let $V$ be a finite-dimensional vector space. Suppose that $L$ is a Lie subalgebra of $\gl(V)$ such that every element of $L$ is a nilpotent transformation of $V$. Then there exists a basis of $V$ in which every element of $L$ is represented by a strictly upper triangular matrix.
\end{thm}

Our strategy to prove Engel's Theorem is to find a vector $v\in V$ such that $x(v)=0$ for all $x\in L$, and then use induction on $\dim V$.

\begin{lem}\label{ad_is_nilpotent}
    Suppose that $L$ is a Lie subalgebra of $\gl(V)$. If $x\in V$ such that the linear map $x:V\to V$ is nilpotent, then $\ad x:L\to L$ is also nilpotent.
\end{lem}

\begin{proof}
    Easy.
\end{proof}

\begin{lem}\label{invariance_lemma_0}
    Suppose that $A$ is an ideal of a Lie subalgebra $L$ of $\gl(V)$. Let 
    \[
        W=\{v\in V:a(v)=0,\;\forall a\in A\}.
    \]
    Then $W$ is an $L$-invariant subspace of $V$.
\end{lem}

\begin{proof}
    For any $v\in W$, $x\in L$ and $a\in A$, since $A$ is an ideal of $L$, we have $[a,x]\in A$. Thus $a(x(v))=x(a(v))+[a,x](v)=0$. Hence $x(v)\in W$.
\end{proof}

\begin{lem}\label{common_zero_eigenvector}
    Suppose that $L$ is a Lie subalgebra of $\gl(V)$, where $V$ is a non-zero finite-dimensional vector space. If every element of $L$ is a nilpotent transformation of $V$, then there is some non-zero $v\in V$ such that $x(v)=0$ for all $x\in L$.
\end{lem}

\begin{proof}
    Use induction on $\dim L$. If $\dim L=1$, then the conclusion is trivial. Now suppose $\dim L>1$. 
    
    We take a maximal Lie subalgebra $A$ of $L$ with $\dim A<\dim L$. We claim that $A$ is an ideal of $L$ and $\dim A=\dim L-1$. Consider the quotient space $L/A$. We define a linear map $\varphi:A\to \gl(L/A)$, $ \varphi(a)(x+A)=[a,x]+A$ for $x\in L$. Verify that $\varphi$ is well-defined and is a Lie algebra homomorphism. Since $a:V\to V$ is nilpotent, by Lemma \ref{ad_is_nilpotent}, $\ad a:L\to L$ is nilpotent, and therefore $\varphi(a)$ is as well. By the inductive hypothesis, there is some non-zero element $y+A\in L/A$ such that $\varphi(a)(y+A)=0$ for all $a\in A$. Hence $[a,y]\in A$ for all $a\in A$. Thus $A\oplus \span\{y\}$ is a Lie subalgebra of $L$. By the maximality of $A$, we have $L=A\oplus\span\{y\}$, and $A$ is an ideal of $L$.

    Now we apply the inductive hypothesis to $A$. Thus there is some non-zero $w\in V$ such that $a(w)=0$ for all $a\in A$. Let $W=\{v\in V:a(v)=0,\;\forall a\in A\}$. Then by Lemma \ref{invariance_lemma_0}, $W$ is $L$-invariant. Hence $y(W)\subset W$. Clearly $y|_W$ is nilpotent, so there exists some non-zero $v\in W$ such that $y(v)=0$. Since $L=A\oplus\span\{y\}$, we know that $x(v)=0$ for all $x\in L$.
\end{proof}

\begin{proof}[Proof of Engel's Theorem]
    We use induction on $n=\dim V$. If $n=1$, then the conclusion is trivial. Now suppose $n>1$. By Lemma \ref{common_zero_eigenvector}, there is a non-zero $v\in V$ such that $x(v)=0$ for all $x\in L$. Let $U$ be the quotient space $V/\span\{v\}$. Any $x\in L$ induces a linear transformation $\bar{x}:U\to U$. Clearly $L\to \gl(U)$ given by $x\mapsto \bar{x}$ is a Lie algebra homomorphism. 

    The image of this homomorphism is a Lie subalgebra of $\gl(U)$, so by the inductive hypothesis there is a basis $\{\bar{v}_1,\ldots,\bar{v}_{n-1}\}$ of $U$ in which the matrices of all $\bar{x}$ are strictly upper triangular. Let $v_i\in V$ be a representative of $\bar{v}_i\in U$, $i=1,2,\ldots,n-1$. Then $\{v,v_1,\ldots,v_{n-1}\}$ is a basis of $V$ in which the matrices of all $x\in L$ are strictly upper triangular.
\end{proof}

\begin{thm}[Engel's Theorem, 2nd version]
    A Lie algebra $L$ is nilpotent if and only if for every $x\in L$, the linear map $\ad x:L\to L$ is nilpotent.
\end{thm}

\begin{proof}
    The "only if" direction is easy. For the "if" direction, consider the adjoint homomorphism $\ad:L\to \ad L\subset \gl(L)$. By the original version of Engel's Theorem, there is a basis of $L$ in which every $\ad x:L\to L$ is represented by a strictly upper triangular matrix. It follows directly that $\ad L$ is nilpotent, so $L$ is nilpotent by Proposition \ref{adL_nilpotent_iff_L_is}.
\end{proof}

\begin{rem}
    Note that Engel's Theorem does \tfs{not} tell us that a Lie algebra $L$ is nilpotent if and only if every element of $L$ is a nilpotent linear transformation of $V$. The "only if" direction is false. For a counterexample, consider the 1-dimensional Lie subalgebra $\span \{\id_V\}$ of $\gl(V)$. 
\end{rem}

\section{Lie's Theorem}

\begin{thm}[Lie's Theorem]
    Let $V$ be a finite-dimensional complex vector space. If $L$ is a solvable Lie algebra of $\gl(V)$, then there exists a basis of $V$ in which every element of $L$ is represented by an upper triangular matrix.
\end{thm}

Note that the Lie algebra $\mathrm{b}(n,F)$ of upper triangular matrices is clearly solvable, so we require that $L$ is solvable in the hypothesis of Lie's Theorem.

Similar to the proof of Engel's Theorem, our strategy to prove Lie's Theorem is to find a common eigenvector $v\in V$ of all $x\in L$, and then use induction on $\dim V$.

\begin{defn}
    A \tfs{weight} for a Lie algebra $L$ of $\gl(V)$ is a linear map $\lambda:L\to F$ such that 
    \[
        V_\lambda:=\{v\in V:x(v)=\lambda(x)v,\;\forall x\in L\}
    \]
    is a non-zero subspace of $V$. The vector space $V_\lambda$ is called the \tfs{weight space} associated to the weight $\lambda$. 
\end{defn}

\begin{lem}[Invariance Lemma]\label{invariance_lemma}
    Assume that $F$ is a field with characteristic 0, and $V$ is a finite-dimensional vector field over $F$. Let $L$ be a Lie subalgebra of $\gl(V)$ and let $A$ be an ideal of $L$. Let $\lambda:A\to F$ be a weight of $A$. Then the weight space 
    \[
        V_\lambda=\{v\in V:a(v)=\lambda(a)v,\; \forall a\in A\}
    \]
    is an $L$-invariant subspace of $V$.
\end{lem}

\begin{proof}
    For every $w\in V_\lambda$, $y\in L$ and $a\in A$, we need to show that $a(y(w))=\lambda(a)y(w)$. Since $a(y(w))=y(a(w))+[a,y](w)=\lambda(a)y(w)+\lambda([a,y])(w)$, we need to show that $\lambda([a,y])=0$.

    For any $0\neq w\in V_\lambda$, consider the subspace $U:=\{w,y(w),y^2 (w),\ldots\}$ of $V$. Clearly $U$ is $y$-invariant. We claim that $\forall z\in A$, the subspace $U$ is $z$-invariant.

    Let $m\geq 1$ such that $\{w,y(w),\ldots,y^{m-1}(w)\}$ is a basis of $U$. Note that $z(w)=\lambda(z)w$, and $z(y(w))=y(z(w))+[z,y](w)=\lambda(z)y(w)+\lambda([z,y])(w)$. More generally, for $k\geq 1$, we have $z(y^{k}(w))=y(z(y^{k-1}(w)))+[z,y](y^{k-1}(w))$. By induction, it is easy to see that $U$ is $z$-invariant, and the matrix of $z|_U$ with respect to the basis $\{w,y(w),\ldots,y^{m-1}(w)\}$ is the upper triangular matrix 
    \[
        \left(
            \begin{array}{cccc}
                \lambda(z)    & * & \cdots & *  \\
                0 & \lambda(z) & \cdots & *           \\
                \vdots & \vdots & \ddots & \vdots \\
                0 & 0 & \cdots & \lambda(z)
            \end{array}
        \right)
    \]
    In particular, the trace of $z|_U$ is $m\lambda(z)$. Take $z=[a,y]$. Since $[a,y]|_U=a|_U y|_U -y|_U a|_U$, we know that $\tr ([a,y]|_U)=0$. But $\tr([a,y]|_U)=m\lambda([a,y])$. So $m\lambda([a,y])=0$. Since $\char F=0$, we have $\lambda([a,y])=0$, completing the proof.
\end{proof}

Now we prove Lie's Theorem.

\begin{proof}[Proof of Lie's Theorem]
    First, we claim that there exists a non-zero vector $v\in V$ such that $v$ is a common eigenvector for all $x\in L$. 

    We use induction on $\dim L$. The case where $\dim L=1$ is trivial. Now assume $\dim L>1$. Since $L$ is solvable, we know that $L'$ is properly contained in $L$. So we may find a subspace $A$ of $L$ such that $A$ contains $L'$, and $L=A\oplus \span\{z\}$ for some $0\neq z\in L$. Clearly $A$ is an ideal of $L$, and $A$ is solvable. Thus by inductive hypothesis, there exists a non-zero $w\in V$ such that $w$ is a common eigenvector for all $a\in A$. Let $\lambda:A\to\mathbb{C}$ be the corresponding weight and let $V_\lambda$ be its weight space. Then by Lemma \ref{invariance_lemma}, $V_\lambda$ is $L$-invariant. Hence there exists a non-zero $v\in V_\lambda$ which is an eigenvector of $z$, and thus $v$ is a common eigenvector for all $x\in L$.

    The remainder of the proof of Lie's Theorem is analogous to the proof of Engel's Theorem, which uses induction on $\dim V$, so we leave this to the reader.
\end{proof}

\section{Cartan's Criteria}

From now on, we only consider finite-dimensional complex vector spaces. We recall the Jordan decomposition of a linear transformation of a finite-dimensional complex vector space. The proof can be easily found in many textbooks on linear algebra.
\begin{lem}[Jordan decomposition]
    Let $x$ be a linear transformation of a finite-dimensional complex vector space $V$. Then there is a unique expression of $x$ as a sum $x=d+n$, where $d:V\to V$ is diagonalizable, $n:V\to V$ is nilpotent, and $d$ and $n$ commute. Moreover, there are polynomials $p(X),q(X)\in\mathbb{C}[X]$ without constant term, such that $d=p(x)$ and $n=q(x)$.
\end{lem}

\begin{cor}
    Let $V$ be a vector space. Suppose that $x\in\gl(V)$ has Jordan decomposition $d+n$. Then $\ad x=\ad d+\ad n$ is the Jordan decomposition of $\ad x:\gl(V) \to \gl(V)$.
\end{cor}

\begin{proof}
    This follows from the uniqueness of the Jordan decomposition.
\end{proof}

Let $L$ be a solvable Lie subalgebra of $\gl(V)$. What does the solvability tell us? The next proposition gives us an indication.

\begin{prop}\label{solvable_subalgebra_tr}
    Let $L$ be a Lie subalgebra of $\gl(V)$. Suppose that $L$ is solvable. Then there exists a basis of $L$ in which every element of $L'$ is represented by a strictly upper triangular matrix. Moreover, $\tr (xy)=0$ for all $x\in L$ and $y\in L'$.
\end{prop}

\begin{proof}
    This follows from Lie's Theorem.
\end{proof}

If $L$ is solvable, then by the proposition above, we can get some information about trace. Conversely, information about trace can also test the solvability of $L$. We have the following proposition:

\begin{prop}\label{tr_L_solvable}
    Let $L$ be a Lie subalgebra of $\gl(V)$. Suppose that $\tr(xy)=0$ for all $x,y\in L$. Then $L$ is solvable.
\end{prop}

\begin{proof}
    To show that $L$ is solvable, it suffices to show that every element of $L'$ is a nilpotent linear transformation of $V$. It will then follow by Engel's Theorem that $L'$ is nilpotent, so $L$ is solvable.

    Take any $x\in L'$. Then $x$ has the Jordan decomposition $d+n$, where $d$ is diagonalizable, $n$ is nilpotent, and $[d,n]=0$. We may fix a basis of $V$ in which $d$ is diagonal and $n$ is upper triangular. Suppose that the diagonal entries of $d$ are $\lambda_1,\ldots,\lambda_m$. Our aim is to show that $d=0$, so it suffices to show that $\sum_{i=1}^{m}\lambda_i\bar\lambda_i=0$.

    Let $\bar{d}$ be the linear transformation on $V$ such that it is diagonal in this basis with diagonal entries $\bar{\lambda}_1,\ldots,\bar{\lambda}_m$. Then by direct computation we have $\tr(\bar{d}x)=\sum_{i=1}^{m}\lambda_i\bar\lambda_i$. Since $x\in L'$, we know that $x$ is a linear combination of commutators in $L$. So we need to show that $\tr(\bar{d}[y,z])=0$ for all $y,z\in L$. Note that $\tr(\bar{d}[y,z])=\tr([\bar{d},y]z)$. By our hypothesis, it suffices to show that $[\bar{d},y]\in L$ for all $y\in L$, i.e. $\ad \bar{d}:\gl(V)\to \gl(V)$ maps $L$ into $L$. 

    Note that the Jordan decomposition of $\ad x:\gl(V)\to \gl(V)$ is $\ad d+\ad n$. Hence there exists a polynomial $p(X)\in\mathbb{C}[X]$ such that $\ad d=p(\ad x)$. It follows from Lagrange interpolation that $\ad \bar{d}$ is a polynomial of $\ad d$. Since $\ad x$ maps $L$ into $L$, we know that $\ad \bar{d}$ maps $L$ into $L$.
\end{proof}

\begin{thm}\label{Cartan_1st_cri}
    Let $L$ be a complex Lie algebra. Then $L$ is solvable if and only if $\tr(\ad x \circ \ad y)=0$ for all $x\in L$ and $y\in L'$.
\end{thm}

\begin{proof}
    Suppose that $L$ is solvable. Then $\ad L\subset \gl(L)$ is a solvable Lie subalgebra of $\gl(L)$. By Proposition \ref{solvable_subalgebra_tr}, we have $\tr(\ad x \circ \ad y)=0$ for all $x\in L,y\in L'$. 

    Conversely, if $\tr(\ad x \circ \ad y)=0$ for all $x\in L$ and $y\in L'$, then by Proposition \ref{tr_L_solvable}, $\ad L'=\{\ad y\in\ad L:y\in L'\}$ is solvable. Since $\ad L'\cong L'/(Z(L)\cap L')$, we know that $L'$ is solvable, and hence $L$ is solvable.
\end{proof}

Now we define the \tfs{Killing form} on a complex Lie algebra $L$. The Killing form on $L$ is a symmetric bilinear form, defined by 
\[
    \kappa(x,y):=\tr(\ad x\circ \ad y)\quad \mbox{for }x,y\in L.
\]
It is easy to verify that $\kappa(x,y)$ is indeed a symmetric bilinear form on $L$. The Killing form has one important property, called associativity. That is, for all $x,y,z\in L$, we have $\kappa([x,y],z)=\kappa(x,[y,z])$. This follows from the identity $\tr([x,y]\circ z)=\tr(x\circ [y,z])$ for $x,y,z\in\gl(V)$.

Using the Killing form, we can restate Theorem \ref{Cartan_1st_cri} as follows:
\begin{thm}[Cartan's First Criterion]
    A complex Lie algebra $L$ is solvable if and only if $\kappa(x,y)=0$ for all $x\in L$ and $y\in L'$.
\end{thm}

Suppose that $L$ is a complex Lie algebra with Killing form $\kappa$. If $I$ is an ideal of $L$, then $I$ itself has a Killing form $\kappa_I$. We show that $\kappa_I$ is equal to the restriction of the Killing form $\kappa$ to $I$.

\begin{lem}
    If $x,y\in I$, then $\kappa_I(x,y)=\kappa(x,y)$.
\end{lem}

\begin{proof}
    Take a basis of $I$ and extend it to a basis of $L$. Then the matrix of $\ad x:L\to L$ in this basis has the form
    \[
        \left(
            \begin{array}{cc}
                A_x   &   B_x  \\
                0     &   0        
            \end{array}
        \right),
    \]
    where $A_x$ is the matrix of $(\ad x)|_I:I\to I$ with respect to the basis of $I$. Hence the matrix of $\ad x\circ \ad y:L\to L$ in this basis of $L$ is 
    \[
        \left(
            \begin{array}{cc}
                A_x A_y   &   A_x B_y \\
                0     &   0        
            \end{array}
        \right).
    \]
    Thus $\kappa(x,y)=\tr (\ad x\circ \ad y)=\tr(A_x A_y)=\kappa_I(x,y)$.
\end{proof}

Next, we introduce Cartan's Second Criterion, which tests whether a complex Lie algebra is semisimple. Recall that a Lie algebra is called semisimple if it has no solvable ideals except 0. Since we can detect the solvability of a Lie algebra by using Killing form, we believe that we can also use the Killing form to test whether a Lie algebra is semisimple.

Let $\beta$ be a symmetric bilinear form on a finite-dimensional complex vector space $V$. For any subset $S$ of $V$, we define the perpendicular space of $S$ to be 
\[
    S^\perp := \{ v\in V : \beta(v,s)=0,\;\forall s\in S \}.
\]
Clearly $S^\perp$ is a subspace of $V$. If $V^\perp=0$, then we say that $\beta$ is \tfs{non-degenerate}. 

If $\beta$ is non-degenerate, then for any subspace $W$ of $V$, by rank-nullity theorem in linear algebra, we have 
\[
    \dim W+\dim W^\perp =\dim V.
\]

\begin{lem}
    Let $L$ be a Lie algebra, and let $I$ be an ideal of $L$. Then $I^\perp$ is also an ideal of $L$. 
\end{lem}

\begin{proof}
    Easy.
\end{proof}

\begin{thm}[Cartan's Second Criterion]
    A complex Lie algebra $L$ is semisimple if and only if its Killing form $\kappa$ is non-degenerate.
\end{thm}

\begin{proof}
    Suppose that $L$ is semisimple. Note that $L^\perp$ is an ideal of $L$. By Cartan's First Criterion, we find that $L^\perp$ is solvable. Hence $L^\perp=0$, which means $\kappa$ is non-degenerate.

    Conversely, if $L$ is not semisimple, then there exists a non-zero abelian ideal $A$ of $L$. Take $0\neq a\in A$. For any $x\in L$, we observe that the composite map $\ad a\circ \ad x \circ \ad a$ sends $L$ to 0. Thus $(\ad a \circ \ad x)^2=0$. Hence $\kappa(a,x)=\tr(\ad a \circ \ad x)=0$ since the trace of a nilpotent linear map is 0. Hence $a\in L^\perp$. Therefore, $\kappa$ must be degenerate.
\end{proof}

Next, we prove that any semisimple complex Lie algebra is a direct sum of simple Lie algebras. 

\begin{lem}\label{semisimple_ideal_oplus}
    Let $L$ be a semisimple complex Lie algebra. Then for any non-trivial proper ideal $I$ of $L$, we have $L=I\oplus I^\perp$. Moreover, $I$ is semisimple.
\end{lem}

\begin{proof}
    Let $\kappa$ denote the Killing form on $L$. It follows from Cartan's First Criterion that $I\cap I^\perp$ is solvable. So $I\cap I^\perp=0$ since $L$ is semisimple. By dimension counting, it follows that $L=I\oplus I^\perp$.

    If $I$ is not semisimple, then by Cartan's Second Criterion, the Killing form on $I$ is degenerate. Hence there exists a non-zero $a\in I$ such that $\kappa(a,x)=0$ for all $x\in I$. But $\kappa(a,y)=0$ for all $y\in I^\perp$. Thus $a\in L^\perp$, and hence $L$ is not semisimple, a contradiction.
\end{proof}

\begin{thm}
    Let $L$ be a complex Lie algebra. Then $L$ is semisimple if and only if there are simple ideals $L_1,\ldots,L_r$ of $L$ such that $L=L_1\oplus\cdots\oplus L_r$.
\end{thm}

\begin{proof}
    Suppose that $L$ is semisimple. We use induction on $\dim L$. Let $I$ be an ideal of $L$ with the smallest possible non-zero dimension. If $I=L$, then we are done. Otherwise $I$ is a proper non-zero ideal of $L$. By Lemma \ref{semisimple_ideal_oplus}, $I$ and $I^\perp$ are semisimple. Hence by inductive hypothesis, we get the desired decomposition.

    Conversely, suppose $L=L_1\oplus\cdots\oplus L_r$, where $L_1,\ldots,L_r$ are simple ideals of $L$. Denote $I=\rad L$. Then $[I,L_i]$ is a solvable ideal of the simple Lie subalgebra $L_i$ for each $i=1,2,\ldots,r$. Hence $[I,L_i]=0$ for each $i$. Thus $[I,L]=0$, which implies that $I\subset Z(L)$. But $Z(L)=Z(L_1)\oplus\cdots\oplus Z(L_r)=0$. Therefore $I=0$, and hence $L$ is semisimple.
\end{proof}

\section{Abstract Jordan Decomposition}

\begin{lem}\label{ss_ad_eq_Der}
    Let $L$ be a finite-dimensional complex semisimple Lie algebra. Then $\ad L=\Der L$. 
\end{lem}

\begin{proof}
    Observe that for any $x,y\in L$ and $\delta\in \Der L$, we have 
    \[
        [\delta ,\ad x](y)=\delta[x,y]-[x,\delta y]=[\delta x,y]=\ad (\delta x)(y).
    \]
    Thus $\ad L$ is an ideal of $\Der L$. Since $L$ is semisimple, the kernel of $\ad:L\to \Der L$ is $Z(L)=0$, so $L\cong \ad L$ and hence $M:=\ad L$ is semisimple.

    Consider the Killing form $\kappa$ on $\Der L$. To show that $M=\Der L$, it suffices to show that $M^\perp=0$. Note that the restriction of the Killing form $\kappa_M$ on $M$ is non-degenerate by Cartan's Second Criterion. It follows that $[M,M^\perp]\subset M\cap M^\perp =0$. Thus if $\delta\in M^\perp$, then $\ad (\delta x)=[\delta ,\ad x]=0$ for all $x\in L$. So $\delta x=0$, and hence $\delta=0$.
\end{proof}

\begin{prop}\label{d_n_in_Der}
    Let $L$ be a complex Lie algebra. Suppose that $\delta\in \Der L\subset \gl(L)$ has a Jordan decomposition $\delta=\sigma+\nu$, where $\sigma$ is diagonalizable, $\nu$ is nilpotent, and $\sigma,\nu$ commute. Then $\sigma,\nu\in \Der L$.
\end{prop}

\begin{proof}
    For $\lambda\in\mathbb{C}$, let 
    \[
        L_\lambda:=\{x\in L: (\delta-\lambda \cdot \id_L)^m x=0\mbox{ for some }m\geq 1\}.
    \]
    Note that if $\lambda$ is not an eigenvalue of $\delta$, then $L_\lambda=0$. By the Primary Decomposition Theorem in linear algebra, we have $L=\oplus_\lambda L_\lambda$. Now we show that $\sigma$ is a derivation, which will imply that $\nu$ is also a derivation. Since $\sigma$ is diagonalizable, its eigenspace of each eigenvalue $\lambda$ is exactly $L_\lambda$. It suffices to show that for any eigenvalues $\lambda,\mu$ of $\delta$, for any $x\in L_\lambda$, $y\in L_\mu$, we have $\sigma([x, y]) = [\sigma(x), y]+[x, \sigma(y)]=(\lambda+\mu)[x,y]$. In other words, it suffices to show that $[L_\lambda,L_\mu]\subset L_{\lambda+\mu}$. This follows from a direct calculation and is left to the reader.
\end{proof}

\begin{thm}[Abstract Jordan Decomposition]
    Let $L$ be a complex semisimple Lie algebra. Then each $x\in L$ can be written uniquely as $x=d+n$, where $d,n\in L$ such that $\ad d$ is diagonalizable, $\ad n$ is nilpotent, and $[d,n]$=0. Furthermore, if $y\in L$ commutes with $x$, then $[d,y]=[n,y]=0$. We say that $x$ has \tfs{abstract Jordan decomposition} $x=d+n$. If $n=0$, then $x$ is said to be \tfs{semisimple}.
\end{thm}

\begin{proof}
    We know that $\ad x\in\gl(L)$ has the Jordan decomposition $\ad x=\sigma+\nu$, where $\sigma\in\gl(L)$ is diagonalizable, $\nu\in\gl(L)$ is nilpotent, and $[\sigma,\nu]=0$. By Proposition \ref{d_n_in_Der} and Lemma \ref{ss_ad_eq_Der}, we know that $\sigma,\nu\in \Der L=\ad L$. So there exist $d,n\in L$ such that $\sigma=\ad d$ and $\nu=\ad n$. Hence $\ad x=\sigma+\nu=\ad d+\ad n=\ad (d+n)$. Since $\ad :L\to L$ is injective, we have $x=d+n$. Moreover, $\ad[d,n]=[\ad d,\ad n]=0$ implies $[d,n]=0$. The uniqueness of $d$ and $n$ follows from the uniqueness of the Jordan decomposition of $\ad x\in\gl(L)$.

    Suppose that $y\in L$ and $[x,y]=0$. Since there is a polynomial $p(X)$ without constant term such that $\nu=p(\ad x)$, it follows that $\nu(y)=0$, i.e. $[n,y]=0$. Hence $[d,y]=0$.
\end{proof}

If the complex semisimple Lie algebra $L$ happens to be a Lie subalgebra of $\gl(V)$, then elements of $L$ have both the abstract Jordan decomposition and the usual Jordan decomposition (as linear transformations on $V$). In fact, the two decompositions agree.

\begin{lem}\label{d_n_in_L}
    Let $L\subset \gl(V)$ be a complex semisimple Lie algebra. For any $x\in L$, let $x=d+n$ denote its usual Jordan decomposition in $\gl(V)$. Then $d,n\in L$.
\end{lem}

\begin{proof}
    See \cite[p. 29]{Humphreys_Lie_alg}. The proof uses Theorem \ref{weyl's_thm} (Weyl's Theorem).
\end{proof}

\begin{thm}
    Let $L$ be a semisimple complex Lie algebra. Let $\theta:L\to \gl(V)$ be a representation of $L$. Suppose that $x\in L$ has abstract Jordan decomposition $x=d+n$. Then $\theta(x)\in\gl(V)$ has Jordan decomposition $\theta(x)=\theta(d)+\theta(n)$ in $\gl(V)$. 
\end{thm}

\begin{proof}
    Since $L$ is semisimple, we have 
    \[
        \ima \theta \cong L / \ker \theta \cong (\ker \theta \oplus (\ker \theta )^\perp) / \ker \theta \cong (\ker \theta)^\perp.
    \]
    Hence $\ima \theta$ is semisimple. Thus we can talk about the abstract Jordan decomposition of elements in $\ima\theta$. Suppose that $x\in L$ has abstract Jordan decomposition $d+n$. By direct computation, we can verify that the abstract Jordan decomposition of $\theta(x)\in\ima \theta$ is $\theta(d)+\theta(n)$. Suppose the usual Jordan decomposition of $\theta(x)\in\gl(V)$ is $\sigma+\nu$. Then by Lemma \ref{d_n_in_L}, we have $\sigma,\nu\in \ima \theta$. Thus $\ad \sigma,\ad \nu:\gl(V)\to \gl(V)$ both map $\ima\theta$ into $\ima\theta$, and they are diagonalizable and nilpotent respectively. So $\ad \theta(x)=\ad \sigma+\ad \nu:\ima\theta\to \ima \theta$ is the Jordan decomposition of $\ad \theta(x)$ in $\gl(\ima \theta)$. Thus $\theta(x)=\sigma+\nu$ is the abstract Jordan decomposition of $\theta(x)$ in $\ima\theta$. By the uniqueness of the abstract Jordan decomposition, we get $\sigma=\theta(d)$ and $\nu=\theta(n)$, completing the proof.
\end{proof}

















\chapter{The Root Space Decomposition and Root Systems}

\section{Representations of $\sl(2,\mathbb{C})$}

Recall that $\sl(2,\mathbb{C})$ is a 3-dimensional Lie algebra with basis 
\[
    x:=
    \left(
        \begin{array}{cc}
            0   &   1 \\
            0   &   0        
        \end{array}
    \right),\quad y:=
    \left(
        \begin{array}{cc}
            0   &   0 \\
            1   &   0        
        \end{array}
    \right),\quad h:=
    \left(
        \begin{array}{cc}
            1   &   0 \\
            0   &   -1        
        \end{array}
    \right).
\]
We can verify that $[x,y]=h$, $[h,x]=2x$, and $[h,y]=-2y$.

One can show that $\sl(2,\mathbb{C})$ is a simple Lie algebra. Hence by Weyl's Theorem, every finite-dimensional $\sl(2,\mathbb{C})$-module is a direct sum of irreducible $\sl(2,\mathbb{C})$-modules. So in this section we will study finite-dimensional irreducible $\sl(2,\mathbb{C})$-modules.

Note that $\ad h:\sl(2,\mathbb{C})\to \sl(2,\mathbb{C})$ is diagonalizable. So $h$ is a semisimple element in $\sl(2,\mathbb{C})$. Let $V$ be any finite-dimensional $\sl(2,\mathbb{C})$-module. For $\lambda\in\mathbb{C}$, set $V_\lambda=\{v\in V:h(v) = \lambda v\}$. Then for all but finitely many $\lambda\in\mathbb{C}$, we have $V_\lambda = 0$. Also $V=\bigoplus_{\lambda\in\mathbb{C}}V_\lambda$.
\begin{lem}
    For all $\lambda\in\mathbb{C}$, we have $x V_{\lambda}\subset V_{\lambda+2}$ and $y V_{\lambda}\subset V_{\lambda-2}$.
\end{lem}

\begin{proof}
    Easy.
\end{proof}

\begin{thm}
    For every $\lambda\in\mathbb{Z}_{\geq 0}$, there exists an irreducible $\sl(2,\mathbb{C})$-module $V(\lambda)$ with basis $\{v_0,\ldots,v_\lambda\}$ such that for each $i=0,1,2,\ldots,\lambda$, we have
    \begin{enumerate}
        \item[(a)] $h(v_i)=(\lambda-2i)v_i$,
        \item[(b)] $x(v_i)=(\lambda-i+1)v_{i-1}$ (set $v_{-1}=0$),
        \item[(c)] $y(v_i)=(i+1)v_{i+1}$ (set $v_{\lambda+1}=0$).
    \end{enumerate}
\end{thm}

\begin{proof}
    Set
    \[
        \bar{x}:=
        \left(
        \begin{array}{ccccc}
                0 & \lambda                          \\
                & 0       & \lambda-1              \\
                &         & \ddots    & \ddots     \\
                &         &           & \ddots & 1 \\
                &         &           &        & 0
            \end{array}
        \right),\quad \bar{y}:=
        \left(
        \begin{array}{ccccc}
                0 &                           \\
                1 & 0       &             \\
                &  2       &  \ddots   &      \\
                &         &      \ddots     & \ddots &  \\
                &         &           &    \lambda    & 0
            \end{array}
        \right),
    \]
    and
    \[
        \bar{h}:=
        \left(
        \begin{array}{cccc}
                \lambda                          \\
                & \lambda-2                     \\
                &           & \ddots            \\
                &           &        & -\lambda
            \end{array}
        \right).
    \]
    Then verify that $[\bar{x},\bar{y}]=\bar{h}$, $[\bar{h},\bar{x}]=2\bar{x}$ and $[\bar{h},\bar{y}]=-2\bar{y}$. Hence this gives a matrix representation of $\sl(2,\mathbb{C})$ with respect to the basis $\{v_0,\ldots,v_\lambda\}$ of $V(\lambda)$. Thus $V(\lambda)$ is an $\sl(2,\mathbb{C})$-module.

    Next, we show that $V(\lambda)$ is irreducible. Let $W$ be any non-zero submodule of $V(\lambda)$. Since $h$ is diagonalizable on $V(\lambda)$, it is also diagonalizable on $W$. Note that all the eigenspaces of $h$ on $V(\lambda)$ are $\span\{v_i\}$, $i=0,1,\ldots,\lambda$, and they are all 1-dimensional. So every eigenspace of $h$ on $W$ must be some $\span\{v_i\}$. Thus $W$ contains some $v_i$. Since $x(v_j)=(\lambda-j+1)v_{j-1}$ and $y(v_j)=(j+1)v_{j+1}$, and $W$ is closed under the action of $x$ and $y$, we know that $W=V(\lambda)$.
\end{proof}

\begin{thm}
    If $V$ is a finite-dimensional irreducible $\sl(2,\mathbb{C})$-module, then $V$ is isomorphic to $V(\lambda)$ for some $\lambda\in\mathbb{Z}_{\geq 0}$.
\end{thm}

\begin{proof}
    Note that $V=\bigoplus_{\mu\in\mathbb{C}}V_\mu$, where $V_\mu=\{v\in V:h(v) = \mu v\}$. Choose $\lambda\in\mathbb{C}$ such that $V_\lambda\neq 0$ and $V_{\lambda+2}=0$. Take $0\neq w_0 \in V_\lambda$. Set $w_i=\frac{1}{i!}y^i(w_0)$ for $i\in\mathbb{Z}_{>0}$. 
    
    Claim that $h(w_i)=(\lambda-2i)w_i$, $x(w_i)=(\lambda-i+1)w_{i-1}$ (where we set $w_{-1}=0$), and $y(w_i)=(i+1)w_{i+1}$. This can be easily verified by using induction on $i$. Hence $\bigoplus_{i\geq 0}\span\{w_i\}$ is a non-zero submodule of $V$, so $V=\bigoplus_{i\geq 0}\span\{w_i\}$. Since $V$ is finite-dimensional, there exists $m\in\mathbb{Z}_{\geq 0}$ such that $w_i\neq 0$ for $i=0,1,\ldots,m$ and $w_{j}=0$ for all $j>m$. Then $(\lambda - m)w_m=x(w_{m+1})=0$, which implies $\lambda=m$. Thus $V=\bigoplus_{i= 0}^{m}\span\{w_i\}\cong V(m)=V(\lambda)$.
\end{proof}



























\section{The Root Space Decomposition}


































\section{Root Systems}















% \include{chapter4}
\backmatter
%\nocite{*}
\bibliographystyle{alpha}
\bibliography{reference}
\end{document}