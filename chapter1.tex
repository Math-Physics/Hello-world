\chapter{Introduction}

Some of the notes come from the lectures given by Professor Chongying Dong in the summer school of Xiamen University in 2019. Other references of the notes are \cite{Humphreys_Lie_alg,Intro_to_Lie_alg}. Most notations in this notes are consistent with \cite{Intro_to_Lie_alg}.

Our goal is to classify all the finite-dimensional complex semisimple Lie algebras up to isomorphism. Unless otherwise stated, all the Lie algebras we discuss are finite-dimensional.

\section{Basic Concepts}

In this section we introduce some basic concepts about Lie algebras and their ideals. We will omit some easy proofs.

\begin{defn}
    A \tfs{Lie algebra} $L$ is a vector space over a field $F$ with a bilinear product $L\times L\to L$, $(x,y)\mapsto [x,y]$, called the Lie bracket, satisfying the following properties:
    \begin{enumerate}
        \item $[x,x]=0$, $\forall x\in L$;
        \item $[x,[y,z]]+[y,[z,x]]+[z,[x,y]]=0$, $\forall x,y,z\in L$, called the \tfs{Jacobi identity}.
    \end{enumerate}
\end{defn}

\begin{rem}
    The condition 1 implies $[x,y]=-[y,x]$, $\forall x,y\in L$, and the converse is also true when $\char F\neq 2$.
\end{rem}

\begin{defn}
    A \tfs{Lie subalgebra} $K$ of a given Lie algebra $L$ is a vector subspace of $L$ such that $[x,y]\in K$ for all $x,y\in K$.
\end{defn}

Here are some examples of Lie algebras.
\begin{enumerate}
    \item Let $\gl(n,F):=M_{n\times n}(F)$ be the vector space of all $n\times n$ matrices over the field $F$ with the Lie bracket defined by $[x,y]:=xy-yx$, where $xy$ is the usual product of matrices $x$ and $y$. Then $\gl(n,F)$ is a Lie algebra, called the \tfs{general linear algebra}.
    \item Let $V$ be a vector space over a field $F$. Then $\gl(V):=\operatorname{End}(V)$ with the Lie bracket $[x,y]:=x\circ y-y\circ x$ for $x,y\in\gl(V)$ is a Lie algebra, also known as the \tfs{general linear algebra}.
    \item Let $V$ be a vector space with the Lie bracket defined by $[x,y]=0$ for all $x,y\in L$. Then this is an \tfs{abelian} Lie algebra structure on $V$.
    \item Let $\sl(n,F)$ be the subspace of $\gl(n,F)$ consisting of all matrices of trace 0. Then $\sl(n,F)$ is a Lie algebra with the same Lie bracket as $\gl(n,F)$, called the \tfs{special linear algebra}.
    \item Let $\mathrm{b}(n,F)$ be the upper triangular matrices in $\gl(n,F)$. Then $\mathrm{b}(n,F)$ is a Lie algebra with the same Lie bracket as $\gl(n,F)$.
    \item Similarly, let $\mathrm{n}(n,F)$ be the strictly upper triangular matrices in $\gl(n,F)$. Then $\mathrm{n}(n,F)$ is a Lie algebra with the same Lie bracket as $\gl(n,F)$.
    \item Let $A$ be an algebra over a field $F$ (i.e. $A$ is both a ring and a vector space over $F$). Let
    \[
        \Der(A):=\{ D\in\operatorname{End}(A):D(ab)=D(a)b+aD(b) \mbox{ for all } a,b\in A \}.
    \]
    Any element in $\Der(A)$ is called a \tfs{derivation} of $A$. We can verify that if $D,E\in\Der A$, then $[D,E]=D\circ E-E\circ D \in \Der A$ (but $D \circ E$ may not be a derivation). Then $\Der(A)$ is a Lie subalgebra of $\gl(A)$.
\end{enumerate}


\begin{defn}
    Let $L$ be a Lie algebra. An \tfs{ideal} of $L$ is a subspace $I$ of $L$ such that $[x,y]\in L$ for all $x\in I,y\in L$ (or equivalently, for all $x\in L,y\in I$).
\end{defn}

An ideal is always a subalgebra, but a subalgebra need not be an ideal. For example, $\mathrm{b}(n,F)$ is a subalgebra of $\gl(n,F)$ but not an ideal when $n\geq 2$.

The Lie algebra $L$ itself is an ideal of $L$, and $\{0\}$ is an ideal of $L$. They are called the \tfs{trivial ideals} of $L$. Another example of an ideal is the \tfs{center} $Z(L)$ of $L$, defined by 
\[
    Z(L):=\{x\in L:[x,y]=0 \mbox{ for all } y\in L\}.
\]
Clearly $Z(L)=L$ if and only if $L$ is abelian.

Now we introduce the definition of homomorphisms between Lie algebras.
\begin{defn}
    Let $L_1$, $L_2$ be two Lie algebras over a field $F$. We say a map $\varphi:L_1\to L_2$ is a \tfs{homomorphism}, if $\varphi$ is a linear map, and 
    \[
        \varphi([x,y])=[\varphi(x),\varphi(y)],\quad \forall x,y\in L.
    \]
    Notice that the first Lie bracket is taken in $L_1$ and the second is taken in $L_2$. We say $\varphi$ is an \tfs{isomorphism} if $\varphi$ is also bijective.
\end{defn}

An extremely important homomorphism is the \tfs{adjoint homomorphism}. If $L$ is a Lie algebra, then we define 
\[
    \ad:L\to \gl(L)
\]
by $(\ad x)(y):=[x,y]$ for $x,y\in L$. Clearly $\ad$ is a linear map, and by Jacobi identity we can check that
\[
    \ad([x,y])=\ad x\circ \ad y-\ad y\circ \ad x=[\ad x,\ad y].
\]
Hence $\ad:L\to \gl(L)$ is a Lie algebra homomorphism. Its kernel is $Z(L)$.

\begin{prop}
    If $\varphi : L_1\to L_2$ is a Lie algebra homomorphism, then $\ker \varphi$ is an ideal and $\ima \varphi$ is a subalgebra.
\end{prop}

\begin{proof}
    This is easy to verify.
\end{proof}

If $I,J$ are ideals of the Lie algebra $L$, then their intersection $I\cap J$, their sum $I+J:=\{x+y:x\in I,y\in J\}$ and their Lie bracket $[I,J]:=\operatorname{Span}\{[x,y]:x\in I,y\in J\}$ are ideals of $L$.

\begin{rem}
    It is necessary to define $[I,J]$ to be the \tfs{span} of commutators $[x,y]$ rather than just the \tfs{set} of commutators. Sometimes the set of commutators is not an ideal. See \cite[Exercise 2.14]{Intro_to_Lie_alg} for an example.
\end{rem}

We write $L':=[L,L]$, called the \tfs{derived algebra} of $L$.

If $I$ is an ideal of the Lie algebra $L$, then the cosets $z+I=\{z+x:x\in I\}$ for $z\in L$ form a quotient vector space $L/I=\{z+I:z\in L\}$. A Lie bracket on $L/I$ can be defined by $[w+I,z+I]:=[w,z]+I$ for $w,z\in L$. We can verify that the Lie bracket on $L/I$ is well-defined, which makes $L/I$ a Lie algebra, called the \tfs{quotient algebra} of $L$ by $I$.


\begin{thm}
    We have some isomorphism theorems for Lie algebras:
    \begin{enumerate}
        \item If $\varphi:L_1\to L_2$ is a homomorphism of Lie algebras, then 
        \[
            L_1/\ker \varphi\cong \ima \varphi.
        \]
        \item If $I$ and $J$ are ideals of a Lie algebra, then $(I+J)/J\cong I/(I\cap J)$.
        \item If $I$ and $J$ are ideals of a Lie algebra $L$ and $I\subset J$, then $J/I$ is an ideal of $L/I$ and $(L/I)/(J/I)\cong L/J$.
    \end{enumerate}
\end{thm}

\begin{proof}
    This is easy to verify.
\end{proof}

\begin{eg}
    Fix a field $F$ and consider the linear map $\operatorname{tr}:\gl (n,F)\to F$, which sends a matrix to its trace. It is easy to see that $\tr$ is a surjective Lie algebra homomorphism, and $\ker\tr =\sl(n,F)$. Then $\gl(n,F)/\sl(n,F)\cong F$.
\end{eg}

\begin{eg}
    Recall that for a Lie algebra $L$, the kernel of the adjoint homomorphism $\ad:L\to \gl (L)$ is the center $Z(L)$. If we denote $\ad L:=\ima \ad$, then $L/Z(L)\cong \ad L$, which is a Lie subalgebra of $\gl(L)$.
\end{eg}

Let $I$ be an ideal of the Lie algebra $L$. There is a bijective correspondence between the ideals of $L/I$ and the ideals of $L$ containing $I$. If $J$ is an ideal of $L$ that contains $I$, then $J/I$ is an ideal of $L/I$. Conversely, if $K$ is an ideal of $L/I$, then $J:=\{z\in L:z+I\in K\}$ is an ideal of $L$ that contains $I$.

For two Lie algebras $L_1,L_2$, we define their \tfs{direct sum} $L_1\oplus L_2:=\{(x_1,x_2):x_i\in L_i\}$ to be the direct sum of their underlying vector spaces with the Lie bracket defined by 
\[
    [(x_1,x_2),(y_1,y_2)]:=([x_1,y_1],[x_2,y_2]).
\] 
Then $L_1\oplus L_2$ is a Lie algebra.

Next we introduce the notion of solvable and nilpotent Lie algebras. 

\begin{lem}
    If $I$ is an ideal of the Lie algebra $L$. Then $L/I$ is abelian if and only if $I$ contains the derived algebra $L'$.
\end{lem}

\begin{proof}
    Clear.
\end{proof}

This lemma tells us that the derived algebra is the smallest ideal of $L$ with an abelian quotient. We define the derived series of $L$ to be the series with terms
\[
    L^{(1)}=L' \quad \mbox{and} \quad L^{(k)}=[L^{(k-1)},L^{(k-1)}] \mbox{ for } k\geq 2.
\]

Then $L\supset L^{(1)}\supset L^{(2)}\supset\cdots$. Each $L^{(k)}$ is an ideal of $L$.

\begin{defn}
    A Lie algebra $L$ is called \tfs{solvable} if for some $m\geq 1$ we have $L^{(m)}=0$.
\end{defn}

\begin{lem}
    Let $L$ be a Lie algebra.
    \begin{enumerate}
        \item If $L$ is solvable, then every subalgebra and every homomorphic image of $L$ are solvable.
        \item If $I$ is an ideal of $L$ such that both $I$ and $L/I$ are solvable, then $L$ is solvable.
        \item If $I$ and $J$ are solvable ideals of $L$, then $I+J$ is also a solvable ideal of $L$.
    \end{enumerate}
\end{lem}

\begin{proof}
    This is easy to verify.
\end{proof}

\begin{cor}
    Let $L$ be a finite dimensional Lie algebra. Then there exists a unique maximal solvable ideal containing every solvable ideal.
\end{cor}

\begin{proof}
    Let $R$ be a solvable ideal of $L$ with the largest possible dimension. Then for every solvable ideal $I$ of $L$, $R+I$ is also a solvable ideal, and $\dim (R+I)\geq\dim R$. Hence $R+I=R$, so $I\subset R$.
\end{proof}

The unique maximal solvable ideal above is called the \tfs{radical} of $L$, and is denoted by $\rad L$.

\begin{defn}
    A non-zero finite dimensional Lie algebra $L$ is called \tfs{semisimple} if $\rad L=0$.
\end{defn}

\begin{prop}
    A Lie algebra is semisimple if and only if every abelian ideal is 0. (Recall that an ideal $I$ of a Lie algebra is called abelian if $[I,I]=0$.)
\end{prop}

\begin{proof}
    Easy.
\end{proof}

\begin{defn}
    A finite dimensional Lie algebra $L$ is called \tfs{simple} if it has no ideals other than 0 and $L$, and $L$ is not abelian.
\end{defn}

\begin{rem}
    Clearly simple Lie algebras are semisimple. The requirement that A simple Lie algebra should not be abelian is to remove the 1-dimensional Lie algebra.
\end{rem}

We will see later that every semisimple Lie algebra is a direct sum of simple Lie algebras, which justifies the name \tfs{semi}simple.

\begin{lem}
    If $L$ is a Lie algebra, then $L/\rad L$ is semisimple.
\end{lem}

\begin{proof}
    Easy.
\end{proof}

This lemma tells us that if we want to understand finite dimensional Lie algebras, it is necessary to understand solvable Lie algebras and semisimple Lie algebras.

Next we introduce the definition of a nilpotent Lie algebra, which is also a very important kind of Lie algebras.

\begin{defn}
    Let $L$ be a Lie algebra. The lower central series of $L$ is defined to be the series with terms 
    \[
        L^{1}=L' \quad \mbox{and} \quad L^{k}=[L,L^{k-1}] \mbox{ for } k\geq 2.
    \]
    Then $L\supset L^{1}\supset L^{2}\supset\cdots$. $L$ is called \tfs{nilpotent} if for some $m\geq 1$ we have $L^m=0$.
\end{defn}

Clearly, nilpotent Lie algebras are solvable, but solvable Lie algebras may not be nilpotent.

\begin{prop}\label{adL_nilpotent_iff_L_is}
    Let $L$ be a Lie algebra.
    \begin{enumerate}
        \item If $L$ is nilpotent, then every subalgebra of $L$ is nilpotent.
        \item $L$ is nilpotent if and only if $\ad L\cong L/Z(L)$ is nilpotent.
    \end{enumerate}
\end{prop}

\begin{proof}
    Easy.
\end{proof}

\section{A Little Representation Theory}

Let $L$ be a Lie algebra over a field $F$. A \tfs{representation} of $L$ is a Lie algebra homomorphism $\varphi:L\to \gl(V)$, where $V$ is a finite-dimensional vector space over $F$. If we fix a basis of $V$, then we obtain a homomorphism $L\to \gl(n,F)$, called the matrix representation of $L$. If $\ker \varphi=0$, then the representation is said to be faithful.

\begin{eg}
    The \tfs{adjoint representation} of a Lie algebra $L$ is 
    \[
        \ad:L\to \gl(L),\quad x\mapsto \ad x.
    \]
    The kernel of the adjoint representation is $Z(L)$.
\end{eg}

\begin{defn}
    Suppose $L$ is a Lie algebra over a field $F$. An $L$-\tfs{module} is a finite-dimensional $F$-vector space $V$, together with a map 
    \[
        L\times V\to V,\quad (x,v)\mapsto x \cdot v
    \]
    satisfying the following properties:
    \[
        (\lambda x+\mu y)\cdot v=\lambda(x\cdot v)+\mu(y\cdot v),\eqno(1)
    \]
    \[
        x\cdot (\lambda v+\mu w)=\lambda(x\cdot v)+\mu(x\cdot w),\eqno(2)
    \]
    \[
        [x,y]\cdot v=x\cdot (y\cdot v)-y\cdot (x\cdot v),\eqno(3)
    \]
    for all $x,y\in L$, $v,w\in V$, and $\lambda,\mu\in F$.
\end{defn}

The representations of a Lie algebra $L$ and $L$-modules are two different ways to describe the same structures.

Let $L$ be a Lie algebra. Suppose $V$ is an $L$-module. A \tfs{submodule} $W$ of $V$ is a subspace of $V$ which is invariant under the action of $L$. 

\begin{eg}
    A Lie algebra $L$ can be viewed as an $L$-module via the adjoint representation. Then the submodules of $L$ are exactly the ideals of $L$.
\end{eg}

If $W$ is a submodule of the $L$-module $V$, then the quotient space $V/W$ is naturally an $L$-module, called the \tfs{quotient module}.

Now we introduce the homomorphism between $L$-modules. Suppose $V,W$ are $L$-modules. An $L$-module \tfs{homomorphism} from $V$ to $W$ is a linear map $\theta:V\to W$ which commutes with the action of $L$. We say $\theta$ is an \tfs{isomorphism} if it is bijective.

The classical isomorphism theorems for usual modules (over a ring) also hold for $L$-modules.

Let $V$ be an $L$-module. We say $V$ is \tfs{irreducible} if it is non-zero and has no submodules other than 0 and $V$. We say $V$ is \tfs{indecomposable} if $V$ cannot be written as the direct sum of two non-zero submodules. We say $V$ is \tfs{completely reducible} if $V$ can be written as the direct sum of some irreducible submodules.

\begin{lem}[Schur's Lemma]
    Let $L$ be a complex Lie algebra. If $S$ is a finite-dimensional irreducible $L$-module, then every $L$-module homomorphism $\theta:S\to S$ is a scalar multiple of the identity transformation, i.e. $\theta=\lambda\cdot\id_S$ for some $\lambda\in\mathbb{C}$.
\end{lem}

\begin{proof}
    Note that $\theta$ must have an eigenvalue, say $\lambda$. Then $\ker (\theta-\lambda\cdot\id_S)$ is a non-zero submodule of $S$, which must be the whole $S$. This completes the proof.
\end{proof}

\begin{thm}[Weyl's Theorem]\label{weyl's_thm}
    Let $L$ be a complex semisimple Lie algebra. Then every irreducible representation of $L$ is completely reducible.
\end{thm}

\begin{proof}
    See \cite[Appendix B]{Intro_to_Lie_alg} or \cite[p. 28]{Humphreys_Lie_alg}.
\end{proof}














